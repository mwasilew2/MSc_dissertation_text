%% Sample chapter file, for use in a thesis.
%% Don't forget to put the \chapter{...} header onto each file.

\chapter{Background}
\bibliography{thesis}

	\section{User Centred Design}

		\subsection{Introduction to User Centred Design}

User Centred Design (UCD) is a broad term that describes both a philosophy and a set of tools used during the design process \citep{norman1986user, norman2013design}. At its core, it gives central role to the needs and limitations of the user. The level of involvement of the user in the design process may vary, but the fundamental difference compared to other approaches is that decisions are driven by a very deep understanding of users’ needs (or even by users themselves). It is not limited to interface optimisation and often means working closely with users already at definition stage where they help in the problem identification. Fundamentally, UCD tries to “focus on usability throughout the entire development process and further throughout the system life cycle” \citep{Gulliksen2003}.

The term User Centred Design was coined and popularized by Donald Norman’s research group in the 1980s. Two influential books were published in that time which he co-authored: “User centered system design” \citep{norman1986user} and “The psychology of everyday things” \citep{norman1988design}.

User Centred Design is sometimes referred to as User Centred System Design (UCSD). This ambiguity comes from the definition of UCD not being agreed upon for many years \citep{Gulliksen2003}.

Concepts behind UCD did not arise in vacuum. The need for “people oriented computers” was already recognized in the early days of computers \citep{ritter2014foundations, nickerson1969man}. Voices of concern were raised that product development methods used at the time were more suitable for big, labour intensive projects and were failing with sophisticated devices which focus on usability \citep{greenbaum1993design, robert1965new}. In 1960s and 1970s there were a number of fields in academia concerned with designing more human friendly devices and processes, but they were applied with varied success. What made UCD so effective was that it “focused on the needs of the user, on activity/task analysis as well as a general requirements analysis, carrying out early testing and evaluation, and designing iteratively.” \citep{ritter2014foundations}. It also emphasized the involvement of the user in the design process instead of treating him purely as a consumer of the product. This has been a paradigm shift that was particularly uncomfortable for managers in the United States who were reluctant to hand over the decision making power \citep{greenbaum1993design}.

UCD has changed over the years.  Initially UCD was focused on command-line tools, but as computers got more widespread and their interfaces became more sophisticated, it started growing in importance and played a different role. With Graphical User Interfaces (GUI) it was focused on layouts and optimisation and with nowadays proliferation of computational systems, UCD design is considering things like personal preferences or social and cultural impact of the device \citep{ritter2014foundations}.

		\subsection{Human Centred Design}

Human Centred Design (HCD) is a broader term that puts humans at the centre \citep{ritter2014foundations, Earthy2001, iso199913407, 1_kurosu_2011}. This means taking into consideration the entire context of the situation in which the product will be used and the human aspects of it. It is considered more interdisciplinary than UCD and is described in many standards \citep{Bevan2001} such as ISO 13407:1999 \citep{iso199913407} and more recently 9241-210: 2010 \citep{dis20099241}. UCD is considered by some as being too much focused on solving a goal-directed, technological problem and limited by considering people solely as users of the system without looking at the organisational goal or counteracting possible adverse effects of use on human health, safety and performance \citep{gasson2003human, gill1996foundations, Bevan2001}. UCD and HCD are not synonyms and HCD does not necessarily imply using UCD methods \citep{Earthy2001, Maguire2001, 1_kurosu_2011, ritter2014foundations}. 
		
		\subsection{Design Driven Innovation}

A recent perspective that is broadening the definition of design to include a reconstructionist \citep{chan2005blue} or social-constructionist \citep{Prahalad2000} view of the market is Design Driven Innovation \citep{Liem2011, verganti2013design}.   

In his book “Design driven innovation: changing the rules of competition by radically innovating what things mean” Roberto Verganti introduces the concept of Design Driven Innovation \citep{verganti2013design}. In his opinion, most organisations understand and use design in two ways: making things beautiful and stylish and having a profound (and thus accurate) understanding of user needs. Innovations coming from these two, beauty of the product and user needs (which is an embodiment of User Centred Design), are in his opinion insufficient for market differentiation and have become so common that they are a norm rather than exception. Verganti argues that what is needed (together with the first two) is a third use for design which is a radical innovation in meaning.

His research reveals that recent management literature focuses on technological innovation and what effect it has on an industry. What is also very well covered is looking beyond features and understanding the meanings behind them - what emotions drive people to buy products. However, the silent assumption is, he continues, that meanings are not a subject of innovation. He proposes a third strategy for design which is innovation in what meaning things can carry.

The author brings and analyses dozens of examples to help better understand design-driven innovation such as:

\begin{itemize}
\item Artemide, Italian lamp manufacturer, created a lamp that is no longer a source of light, but an object that has influence on people’s mood. Effectively, by providing a device that can change intensity and colour of the light you are enabling people to control their mood and the product becomes an element of well-being.

\item The MP3 players were present before iTunes, but it was a change in how to think about music brought by Steve Jobs that revolutionised the industry. Many executives and lobbing groups stubbornly focused on enforcing copy-protection, whereas Apple enabled users to buy a single song instead of an entire album, taste and mix music, create personal playlists.

\item Anthropomorphism in the shape of kitchen appliances brought by Alessi, turned equipment into objects of affection, things you bond with, “teddy bears for adults” \citep{verganti2013design}.

\item Apple’s move to release a notebook without an optical drive was considered a bold one, but Steve Job had an understanding of what cloud computing and wireless connectivity meant – constant access to vast amounts of data and thus no use for CDs/DVDs.
\end{itemize}

The author also provides a structured framework for thinking about innovation in meaning and deploying it in an organisation. Design Driven Innovation extends beyond User Centred Design, but does not discredit it.

	\section{Data-driven design}
	
Data-driven design is an emerging field of study that gained popularity with the digitization of our world and in particular with what is known as big data. The premise of data-driven design is an additional layer of perception provided by data collection and processing, previously unavailable to humans. Although the practices of data-driven design are far from being well established, more and more voices are being raised that consider it a very viable tool when used properly with other methods \citep{Neirotti2014}.
	
		\subsection{What would a cup say if it could speak?}
		
			\subsubsection{What is big data?}



adf


	\section{Double Diamond}