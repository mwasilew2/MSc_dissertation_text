%% Sample chapter file, for use in a thesis.
%% Don't forget to put the \chapter{...} header onto each file.

\chapter{Introduction}

Over the last few years, the School of Informatics has been collaborating with the City of Edinburgh Council in the area of open data in initiatives such as the Smart Data Hack and the Council's EdinburghApps hackathons. In the context of Edinburgh Living Lab, this relationship has broadened into investigating other areas of data science, and new kinds of collaboration. My MSc project is taking place within this context, and is focussing on bringing analytic techniques to bear on Customer Relationship Management (CRM) data that has been collected by the Council over the last year.

	\section{Context}

As one of the fastest growing local authority areas in Scotland, Edinburgh is facing an ever increasing demand for Council services, outstripping the funds available to meet this demand. There are a number of projects on-going in the Council that try to address the resulting challenges, one of which aims to improve the way that Council interacts with residents, particularly in terms of dealing with complaints and reports of problems. At the moment, citizens can communicate with the Council using multiple "channels": email, web forms, mobile apps, phone, post and face-to-face conversation. So-called "Channel Shift" is the policy of encouraging residents to use web forms in preference to other communication channels. Some other objectives include informed design of interfaces and web-forms, increase in the use of digital channels and decrease in traditional channels for selected transactions. The Council has been recently building capacity to collect data and use sophisticated tools for managing and integrating it. This project is hoping to contribute to internal resources for extracting business insights from analysing this data. More broadly, I hope that my research will help the Council to ensure that transactions initiated via digital channels are dealt with effectively, as well as contribute to creating "success stories" and know-how within the Council.

	\section{Objective of the project}

Using analysis of CRM data provide insights about the delivery of CEC services to the residents of Edinburgh. These insights should serve as guidelines for improvement of existing interactions between the Council and citizens as well as help in implementation of transactions for services which are not supported over digital channels yet.

	\section{Thesis structure}

The first part of this thesis is devoted to providing a theoretical background to the work undertaken. User Centred Design (UCD) is a concept in design that has played a major role in building interfaces to computational systems over the last three decades. It encompasses a set of tools and practices used when designing interfaces and services (e.g. web-form, application interface). It is described here providing a historical context and modern developments in related fields. Human Centred Design, which is a broader term, not only considers the person as a user (as it is in the case of UCD), but also as a human. Design Driven Innovation was proposed by Roberto Verganti and it is expanding design strategies and methods (e.g. UCD) to include innovation in meaning. Data-driven design is a practice of designing with the use of data driven rather than human driven (ethnographical) methods. Double Diamond methodology is a model of practicing design (conducting design related activities) which is claimed to be describing a universal framework for a design process, not limited to any particular field.

The second part is describing the work undertaken and is divided into four phases in accordance to the Double Diamond model. The Discover phase describe the initial, exploratory activities in the project. The Define phase narrow down the scope, finishing with three defined objectives. The Develop stage is describing the process of implementation and results. The Deliver part covers delivering the results to the clients and feedback received.

The last two parts are dedicated to evaluating the project and drawing conclusions. Tools used in the process are evaluated, the work undertaken and methodology used. 