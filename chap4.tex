%% Sample chapter file, for use in a thesis.
%% Don't forget to put the \chapter{...} header onto each file.

\chapter{Analysis and evaluation}

	\section{Evaluation of the tools used}

		\subsection{CRM data}
		
		\subsection{IBM Cognos}

	\section{Evaluation of the work undertaken}

		\subsection{Report 1}
		
		\subsection{Report 2}
		
		\subsection{Report 3}
		
	\section{Evaluation of methodology used}

The double diamond approach seems to reflect very well the dynamism of real life projects. The model describes “a rhythm of activities” that comes naturally. It includes a very open, exploratory first stage which leaves space for flexibility in adapting to what would be useful to the client.

The Discovery phase was extremely helpful in understanding the context of the problem and establishing ground before the next phases. Having such an open attitude requires a lot of persistence. The responsibility for the entire project rests on the designer and this causes “a creative stress”. In the early stages, it is desired to not be limited by having a concrete idea of what to do in the project (which is not synonym with not having a path of action). The designer is exposing himself to the unknown and at many points the project could completely change direction or a path could be closed unexpectedly. It is critical to maintain composure, agility and be able to quickly adjust to the new conditions. It is also important to mention that it exposes the project to the will of people across the entire organisation. The more the involved people are open and willing to help the better the outcome of the project will be. In terms of this particular project, the Council employees were very helpful and open-minded and their support has helped tremendously.

The three objectives that came from the Define phase (design brief) were developed in close cooperation with the beneficiaries (and at the same time the requesting party).

The Develop phase managed to address all questions from the previous stage. Having clear, measurable objectives, which were thought through, helped in planning the rest of the process and designing the technical aspects of it. The extent to which implementations were able to solve those problems was described in sections above (Evaluation of work undertaken).

The key outcome of Deliver stage was feedback from clients. It was very helpful to understand the extent to which it addressed actual needs and whether it succeeded in contributing to the on-going efforts in the Council (being in line with the current ICT strategy at the Council).