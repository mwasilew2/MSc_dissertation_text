%% Sample chapter file, for use in a thesis.
%% Don't forget to put the \chapter{...} header onto each file.

\chapter{Conclusion}

Out of this project come many open questions and potential for further study. This dissertation gives a lot of details about the context of the project which were not available before. They can be of significant help in future endeavours.

The double diamond approach was particularly good at enabling cross department activity and flexibility in adjusting the scope of the project to the needs of the Council. Given the experience gained, a further study could try and evaluate different methods used at each stage of the process.

It is also important to stress that such projects are agile in nature and depend heavily on the organisation in which they are run, i.e. on the knowledge of people involved and their willingness to share it. This project is a great example of how open-mindedness of employees can help.

The timescale of the project was extremely short given its complexity. Many parts of it could easily take months to be properly developed. However, it was not aiming at delivering a fully-fledged product. Instead, the objective was to help the Council with evaluating new ways of thinking and working and looking at the design process in its entirety. As a result in many cases compromises had to be made.

BI reports like the ones generated in this project, should be treated as part of a bigger "transformation" project. Identifying cases where users struggled with a web interface by CRM data analysis should be one of many tools in the repertoire of a service manager. For example, they can be used to identify the demographics of people to invite for participation in a focus group.

Reports like these often raise further questions, e.g. when conducting analysis other things start emerging which could be objects of investigation themselves. There is a vast amount of possible insights coming from the CRM data.

The reports can be used by CEC employees on other data sets (it is a matter of pointing to a different source file). 

Coming up with insights and recommendations is only one step. Another question is how to manage a change in an organisation in order to benefit from those analyses. Ability to adapt to user needs and learn from the feedback is actually executed when such insights are followed by tangible actions such as a decision to deploy a change or a confirmation that current efforts are not misplaced.