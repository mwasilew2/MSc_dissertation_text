\chapter{List of files created during the project}

Files created up to Development Phase:
\begin{itemize}
\item “MW v0.1 self” - file to learn Cognos, shows how to combine CRM data with Mosaic data
\item “MW v0.1 self2” - first attempt to create a chart, doesn't work
\item “MW v0.1 self3” - attempt to create a chart, the query itself is working (everything is correct from the sytax point of view), but data is unavailable (aggregate function? filtering error?)
\item “MW v0.1 self4” - debugging the self3 report, added a list to show filtered data returned from the query. Result: there are some entries after filtering, but not as expected
\item “MW v0.1 self5” - from the start, this time the results are as expected (type of result, number of result)
\item “MW v0.2, working chart, 3 days, type of report – day” - first chart working as expected, it shows numbers about queries from all services grouped into Mosaic groups, counted by reference, limited to 3 days (x - mosaic group, e.g. B, G, K; y - count by reference, multiple columns - different values in 'subject')
\item “MW v0.2, working chart, entire May (previous was 4 days), Mosaic group – subject” - second chart working as expected, more accurate; x - type code, e.g. A01, A02; y - count by reference; only 4 services - 4 columns; for the entire May
\item “MW v0.2, working chart, entire May, Mosaic group – subject” - similar to previous one, not as detailed, x - Mosaic groups, e.g. B, G; y - count by reference; columns - different services (different values in 'subject'); entire May
\end{itemize}

Files created during the Development phase:
\begin{itemize}
\item “MW one issue on multiple channels v1.0” - find cases of misuse of multiple channels
\item “MW report 3”, “MW Report 3, blank chart” - base report that can be used to generate different reports about number of interactions with the Council, couldn't overwrite the original one, created another one to have a more meaningful name
\item “MW report 3, chart 1” - both above and below 3 interactions on one chart
\item “MW report 3, chart 1.1 blank” - template for the charts 1.1 and 1.2
\item “MW report 3, chart 1.1” - chart with only above 3 interactions, using filter
\item “MW report 3, chart 1.1 correct” - chart with only above 3 iteractions, using filter and a list to show data (identical to the previous one, extended with the list, couldn't overwrite the previous one)
\item “MW report 3, chart 1.2 correct” - chart with only below 3 interactions, with list
\end{itemize}

the last 3 charts did not have axis sorted, I figured out how to sort axis starting from chart 3.2
 
further parts of report 3 has 2 dimensions which have many entries. As a result it would have a lot of columns and would not be easily readible. So I decided to split it into 3 charts: groups x services; detailed groups x services; selectable service -> group

\begin{itemize}
\item “MW report 3, chart 2.1” - Mosaic group x services
\item “MW report 3, chart 2.2” - Mosaic group detailed x services
\item “MW report 3, chart 2.3 group code” - the use of the selected service across different Mosaic groups
\item “MW report 3, chart 2.3 group type code” - the use of the selected service across different Mosaic detailed groups
\item “MW report 3, chart 3.1” - blank report that is used as a basis for the other 3.1 reports
\item “MW report 3, chart 3.1 above 3” - Mosaic groups that active users belong to
\item “MW report 3, chart 3.1 up to 3” - Mosaic groups that occassional users belong to
\item “report 3.1 “- 3 charts on one page
\end{itemize}

working 27.07
 
\begin{itemize}
\item “MW Report 2, 2.1” – identify citizens who interacted multiple times
\item “MW Report 2, 2.2” – multiple interactions through different channels
\end{itemize}

Recommendations for further analysis after 2.2: analyze the issues, what happened there, comments, any unstructured data around this issue that is available in the system, closed times and dates, when this person interacts with the council again ask them questions what was the problem.
 
The result might be that only people with multiple issues will be contacted, this does not have to be the case. It should be compared with no feedback at all and in this case you can identify people who could provide feedback with very high accurracy. As a counterbalance, the questions might be asked to people from the same social group.
 
Analyze a few examples:
\begin{itemize}
\item Someone starts on the web, the next day they call 
\item Someone start with face-to-face, then they open a number of tickets on the same issue through web the next day
\end{itemize}